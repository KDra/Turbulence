\section*{Question 10}
\textit{Based on everything you have seen from the results, what can you say about the nature of flows in “flow1” and “flow2”.}

\textsl{Q1} shows that statistical data alone is not sufficient to describe the nature of a flow since both have the same values (see \ref{tbl1}). \textsl{Q2} starts showing the difference between the two flows: the $du/dx$ PDF of the first flow has a very wide with a kurtosis similar to that of a normal distribution. This would suggest that the flow is not fully turbulent as in the second case, where the high amount of mixing forces the derivative PDF to be highly leptokurtic and thus have a higher agglomeration around the mean.

\textsl{Q3} shows the energy dissipation computed through statistical methods so that it can be compared to that obtained through spectral methods in \textsl{Q9}. This was, however, not achieved with the differences being by several orders of magnitude higher in \textsl{Q3}. The reason for this is unknown and even after numerous attempts at solving this discrepancy, no solution was found neither in the lecture slides nor in other sources\cite{burchard01}\cite{george2009lectures}\cite{wilcox1998turbulence}.

\textsl{Q5} examines the autocorrelation for the two flows and as is expected the first flow displays $0$ autocorrelation as it would be expected of a laminar or transitioning flow, where structures do not exist to move with the flow. In \textsl{Q6} the scales related to turbulent flow are computed statistically and then compared to spectral methods in \textsl{Q8}, where a short discussion is made around the differences.

The conclusion of this exercise is that the statistics of a laminar and turbulent flow can look very similar or they can be identical but when examined around using other physical parameters the distinctions become clear. Spectral methods seem to be a more reliable for differentiating between flows and they make the computation of other physical parameters easier and computationally more efficient through the use of Fourier transforms.