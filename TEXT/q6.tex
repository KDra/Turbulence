\section*{Question 6}
\textit{Using the autocorrelation function, calculate the Integral length scale and the Taylor micro scale for both datasets. Comment on the results obtained.}

The values in table \ref{scales} are obtained using eq. \eqref{int-l} and \eqref{lambda}, where $\rho$ is the autocorrelation coefficient and $\tau$ the time. To obtain eq. \eqref{lambda} a second degree central difference scheme with a fourth order of accuracy was used at $\tau = 0$. The chosen window size is of $1s$ and the integral scale is given in meters (i.e. the acquisition signal mutiplied by eq. \eqref{int-l} and the mean velocity $\langle u \rangle$).

\begin{equation}\label{int-l}
T_{int} = \int_0^{\infty} \rho (\tau) d\tau
\end{equation}

\begin{equation}\label{lambda}
\lambda = \sqrt{- \frac{2}{\rho''(0)}}
\end{equation}

\begin{table}[!ht]
\centering
\caption{Integral scale($T_{int}$) and Taylor microscale($\lambda$) for flows 1 and 2}
\label{scales}
\begin{tabular}{c|c|c}
Flow & $T_{int}$ & $\lambda$ \\
\hline
1 & $9.44 \times 10^{-5}m$ & $1.49 \times 10^{-5}$\\
2 & $6.53 \times 10^{-2}m$ & $1.51 \times 10^{-4}$
\end{tabular}
\end{table}

Table \ref{scales} shows that in the second flow case the energy carying eddies are two orders of magnitude larger than in the first case. The Taylor microscale is an indicator of the eddy scale at which te turbulent energy is dissipated trough viscosity.