\section*{Question 6}
\textit{Using the autocorrelation function, calculate the Integral length scale and the Taylor micro scale for both datasets. Comment on the results obtained.}

The values in table \ref{scales} are obtained using eq. \eqref{int-l} and \eqref{lambda}, where $\rho$ is the autocorrelation coefficient and $\tau$ the time. To obtain eq. \eqref{lambda} a second degree central difference scheme with a fourth order of accuracy was used at $\tau = 0$. The chosen window size is of $1s$ and the integral scale is given in seconds (i.e. the acquisition signal mutiplied by eq. \eqref{int-l}).

\begin{equation}\label{int-l}
T_{int} = \int_0^{\infty} \rho (\tau) d\tau
\end{equation}

\begin{equation}\label{lambda}
\lambda = \sqrt{- \frac{2}{\rho''(\tau)}}
\end{equation}

\begin{table}[!ht]
\centering
\caption{Integral scale($T_{int}$) and Taylor microscale($\lambda$) for flows 1 and 2}
\label{scales}
\begin{tabular}{c|c|c}
Flow & $T_{int}$ & $\lambda$ \\
\hline
1 & $7.46 \times 10^{-6}s$ & $1.49 \times 10^{-5}$\\
2 & $6.62 \times 10^{-3}s$ & $1.51 \times 10^{-4}$
\end{tabular}
\end{table}

Table \ref{scales} shows that in the second flowcase the velocity is correlated with itself by three orders of magnitude longer than in the first. The Taylor microscale is a description of how often a signal passes $0$.